\documentclass[11pt, a4paper]{article}

\usepackage{minimal/stylesheet}

\institution{Institution}
\project{Project}
\title{Title of the report}
\author{Fabio Matti}
\supervisor{Supervisor}
\date{\today}

\begin{document}

\maketitle

\section{Test section}
\label{sec:label}

This is just a \emph{test} section.

\subsection{Gauss-Bonnet Theorem}
\label{subsec:gauss-bonnet}

We now state a very remarkable result from differential
geometry \citep{article2021}:

\begin{block}{Gauss-Bonnet Theorem}

    Suppose $S$ is a regular surface with Euler characteristic $\chi$.
    It then holds that

    \begin{equation}
        \int_S K~dA = 2\pi\chi + \sum_i \phi_i
        \label{equ:gauss-bonnet}
    \end{equation}

\end{block}

\subsubsection{Consequences}
\label{subsubsec:gauss-bonnet}

This theorem implies the following consequences:

\begin{itemize}
    \item The sphere has a total curvature of $4\pi$.
    \item A plane has identical zero curvature.
\end{itemize}

\subsection{Floats}
\label{subsec:floats}

\hyperref[fig:test]{Figure \ref*{fig:test}} and \hyperref[tab:test]{Table \ref*{tab:test}}
show samples of the theme.

\begin{table}[h]
    \caption{Neatly formatted table}
    \label{tab:test}
    \centering
    \renewcommand{\arraystretch}{1.2}
    \begin{tabular}{@{}lcc@{}}
        \toprule
        Name & Age & Height \\
        \midrule
        Fabio Matti & 22 & 181 \\
        \bottomrule
    \end{tabular}
\end{table}

\begin{figure}[h]
    \centering
    \begin{tikzpicture}
        \fill[lightblue] (0, 0) circle (0.5);
        \fill[darkblue] (0.5, 1) circle (1);
    \end{tikzpicture}
    \caption{Beautiful modern art}
    \label{fig:test}
\end{figure}

\bibliography{biblio.bib}

\end{document}